\documentclass[cs4size,a4paper,adobefonts,openany]{ctexbook}
\usepackage[colorlinks=true,linkcolor=black]{hyperref}
\usepackage{indentfirst}
\usepackage[a4paper,left=2.5cm,right=2.5cm,bottom=2.5cm,top=2.5cm]{geometry}
\usepackage{fontspec}
\setmainfont{Minion Pro}
\pagestyle{plain}
\punctstyle{kaiming}
%\usepackage{unicode-math}
%\setmathfont{STIX Math}

\usepackage{amsmath,amsthm,amssymb}
\newtheorem{defn}{定义}
\newtheorem{thm}{定理}
\newcommand{\pname}[1]{\underline{#1}}
\numberwithin{equation}{section}
\CTEXsetup[number=\thechapter]{chapter}
\begin{document}
\title{\bfseries Regular Polytopes 的学习笔记}
\author{王盛颐}
\date{}
\maketitle
\setcounter{page}{1}
\chapter{POLYGONS AND POLYHEDRA}
\section{Regular polygons}
\begin{defn}
  假设有 $p$ 个点 $A_1,A_2,\dots,A_p$,那么 $p$-边形($p$-gon)被定义为
  连接这些点的 $p$ 条直线段 $A_1A_2,A_2A_3,\dots,A_pA_1$ 形成的环路。这
  些线段和点分别被称为多边形的\pname{边}和\pname{顶点}。
\end{defn}
\end{document}
